\chapter{Ideen und Konzepte}

%TODO: Beispieltext löschen und durch eigenen Text ersetzen.

Hier geht es um die Fragestellung, wieSie die formulierten Ziele der Arbeit erreichen wollen. Sie halten z.B. erste, grobe Ideen, skizzenhafte Lösungsansätzefest. Gibt es mehrere Wege, Ansätzeum dieses Ziel zu erreichen, begründen Sie hier, warum Sie einen bestimmten Weg einschlagen. Beispiel für ein Softwareprojekt: Erste Gedanken über einegrobe Systemarchitektur. Ist z.B. eine Microservice-Architektur angebracht? Welche Alternativen bestehen, wo gibt es Problempunkte? Die Umsetzung, die Beurteilung der Machbarkeit und die detaillierte Beschreibung der umgesetzten Architektur sinddann Teil der Realisierung. Abgrenzung zu Kapitel 5:-Besteht ein wesentliches Projektziel darin, für Ihre Kunden z.B. ein Security-Konzept, ein Kommunikations-Konzeptes, ein IT-Fachkonzept oder einanderes Fach-Konzeptzuerstellen, dann wird die Entwicklungdieser(fachlichen) Konzepte unter «Realisierung» beschrieben(sie sind ja der eigentliche Kern Ihrer Arbeit).-Besteht z.B. ein wesentliches Ziel der Arbeit darin, eine passende Software-Architektur zuevaluieren, dann gehören die entsprechenden Beschreibungen ins Kapitel 5.