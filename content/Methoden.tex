\chapter{Methoden}

%TODO: Beispieltext löschen und durch eigenen Text ersetzen.

Hier halten Sie festund begründen, welches Vorgehensmodell Sie für Ihr Projekt wählen.Sie verweisen allenfalls auf die daraus entstandenen, konkreten Terminpläne mit Meilensteinen, welche z.B. unter Realisierung (Kapitel 5) oder im Anhang versorgt sind.Bei Projekten mit einer verlangten wissenschaftlichen Tiefe werden hier die geplanten Forschungsmethodenwie quantitative/qualitative Interviews, Befragungen, Beobachtungen, Feldexperimentetc. beschriebenund begründet. Warum ist in Ihrer Situation ein Interview besser als eine Umfrage? Wer soll interview werden?  
3(Sie können bei Bedarf in Absprache mit Ihrer Betreuungsperson dazu auch ein zusätzliches Methodencoaching beziehen).Bei Engineering-Projekten halten Sie weitere einzusetzende fachliche Methoden oder Techniken fest.Bei einem Softwareprojekt können dies z.B. der geplante Einsatz einer Anforderungsanalyse, der Einsatz vonReview-Techniken (Architektur-Reviews) oder bekannter Programmiertechniken sein. Dazu gehört auch eine Teststrategie (wo setzen Sieim Projekt Schwerpunktebetr. Testen?). Die eigentliche Testdurchführung ist dann unter Realisierung, im Anhang oder einemselbstständigen Testdokument beschrieben.