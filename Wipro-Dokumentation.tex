%-----------------------------
% Master LaTeX Dokument
%-----------------------------

%-----------------------------
% Konfiguration des Dokuments
%-----------------------------

\documentclass[a4paper,11pt]{scrreprt}

%-----------------------------
% Installierte Pakete
%-----------------------------

\usepackage[ngerman]{babel}				% Rechtsschreibekorrektur Deutsch https://ctan.org/pkg/babel

\usepackage[utf8]{inputenc}				% Unicode Support https://ctan.org/pkg/inputenc

\usepackage[a-2b,latxmp]{pdfx}				% PDFA Support https://ctan.org/pkg/pdfx
\catcode30=12						% What is code without a workaround? See https://tex.stackexchange.com/questions/564990/solved-error-after-miktex-reinstall-text-line-contains-an-invalid-character

\usepackage{graphicx}					% Einfügen von Bildern https://ctan.org/pkg/graphicx?lang=en

\usepackage{tabularx}					% Erweiterte Optionen für Tabellen https://ctan.org/pkg/tabularx

\usepackage{enumitem}					% Aufzählungen https://ctan.org/pkg/enumitem

\usepackage{caption}					% Erweiterte Optionen für Captions https://ctan.org/pkg/caption

\usepackage{pdfpages}					% PDF Dokumente einbetten https://ctan.org/pkg/pdfpages

\usepackage{fancyhdr}					% Für custom Kopf- und Fusszeilen
\pagestyle{fancy}

\usepackage[acronym]{glossaries}		% Für die Erstellung eines Glossars, wir rufen glossaries ohne [toc] auf weil sonst das Abkürzungsverzeichnis im Inhaltsverzeichnis erscheint und die HSLU das nicht möchte.
\loadglsentries{Glossar.tex}			% Das ausgelagerte Glossar wird aus der Datei "Glossar.tex" geladen.

\makeglossaries

\usepackage{amssymb}

\usepackage[T1]{fontenc}				% Fontencoding https://www.ctan.org/pkg/fontenc

\usepackage[backend=biber,style=apa]{biblatex}		% Zitierung nach APA Standard
\addbibresource{Referenzen.bib}

\usepackage{hyperref}					% Formatieren von Links

%-----------------------------
% Inhalt des Dokuments
%-----------------------------

\begin{document}
	
\fancyfoot[OR]{\thepage}

\pagenumbering{gobble}


\begin{titlepage}
	\centering
	\includegraphics[width=0.15\textwidth]{example-image-1x1}\par\vspace{1cm}
	{\scshape\LARGE Hochschule Luzern \par}
	\vspace{1cm}
	{\scshape\Large Wirtschafsprojekt\par}
	\vspace{1.5cm}
	{\huge\bfseries Wie bekomme ich den Bachelor ohne Zusatzaufwand\par}
	\vspace{2cm}
	{\Large\itshape Br. Uno Joho\par}
	\vfill
	betreut durch\par
	Dr. phil ~Roger \textsc{Hämmerli}
	
	\vfill
	
	% Bottom of the page
	{\large \today\par}
\end{titlepage}

\newpage

\noindent
\fontsize{12}{14}
\textbf{Wirtschaftsprojekt an der Hochschule Luzern -- Informatik} \\ \vspace*{0.6cm}

\fontsize{10.5}{12}
\noindent
\textbf{Titel:} \\ \vspace*{0.2cm}

\noindent
\textbf{Studentin/Student:} \newline \newline
\textbf{Studentin/Student:} \newline \newline
\textbf{Studiengang:} BSc Informatik oder Wirtschaftsinformatik  \newline \newline
\textbf{Jahr:} \newline \newline
\textbf{Betreuungsperson:} \newline \newline
\textbf{Expertin/Experte:} \newline \newline
\textbf{Auftraggeberin/Auftraggeber:} \newline \newline \newline
\textbf{Codierung / Klassifizierung der Arbeit:}\\
$\square$ A: Einsicht 	(Normalfall) \\
$\square$ B: R\"ucksprache	(Dauer: \ \ \ \ \        Jahr / Jahre)\\
$\square$ C: Sperre	(Dauer: \ \ \ \ \        Jahr / Jahre)\\


%%% you can use \boxtimes for filling a cross inside the square
%%% e.g., $\boxtimes$ A: Einsicht 	(Normalfall) 


\paragraph{\textbf{Eidesstattliche Erkl\"arung}}
Ich erkl\"are hiermit, dass ich/wir die vorliegende Arbeit selbst\"andig und ohne unerlaubte fremde Hilfe angefertigt haben, alle verwendeten Quellen, Literatur und andere Hilfsmittel angegeben haben, w\"ortlich oder inhaltlich entnommene Stellen als solche kenntlich gemacht haben, das Vertraulichkeitsinteresse des Auftraggebers wahren und die Urheberrechtsbestimmungen der Fachhochschule Zentralschweiz (siehe Merkblatt \flqq Studentische Arbeiten\frqq\ auf MyCampus) respektieren werden. \newline \newline
Ort / Datum, Unterschrift	\underline{\hspace*{8cm}} \newline \newline
Ort / Datum, Unterschrift	\underline{\hspace*{8cm}} \newline \newline \newline
\textbf{Ausschliesslich bei Abgabe in gedruckter Form: \\
Eingangsvisum durch das Sekretariat auszuf\"ullen}\newline \newline
Rotkreuz, den	\underline{\hspace*{4cm}}	\hspace*{1cm} Visum:	\underline{\hspace*{4cm}}

\newpage

\pagenumbering{roman}

\begin{abstract}
Hier steht ein Beispieltext. \\
Zudem wird auf den Glossareintrag zu Wipro verwiesen. \gls{wipro}

Tastaturen werden typischerweise über \gls{usb} angeschlossen. Nicht aber an der \gls{hslu}

Ah doch, jetzt sehe ich das sie an der \gls{hslu} auch \gls{usb} verwenden.

Lorem ipsum dolor sit amet~\autocite{einstein}.
    At vero eos et accusam et justo duo dolores et ea rebum~\autocite{einstein}.

% TODO: Hier den Beispieltext löschen und eigenen Abstract schreiben
\end{abstract}

\tableofcontents

\printglossary[type=\acronymtype, title=Abkürzungsverzeichnis]

\clearpage

\pagenumbering{arabic}

\fancypagestyle{plain}{%
	\renewcommand{\headrulewidth}{0pt}%
	\fancyhf{}%
	\fancyfoot[R]{\thepage}%
}

\input{content/Einleitung.tex}

\input{content/Stand_der_Technik}

\input{content/Ideen_und_Konzepte}

\input{content/Methoden}

\input{content/Realisierung}

\input{content/Evaluation_und_Validation}

\input{content/Ausblick}

\appendix

\printglossary[style=altlist,title=Glossar]
\addcontentsline{toc}{chapter}{Glossar} 		% Macht das "Glossar" im Inhaltsverzeichnis erscheint.

\listoffigures

\listoftables

\printbibliography

\end{document}
